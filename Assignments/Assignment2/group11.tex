\documentclass[]{article}
\usepackage[english]{babel}
\usepackage[utf8x]{inputenc}
\usepackage{amsmath}
\usepackage{graphicx}
\usepackage{amssymb}
\usepackage[left=2cm,right=2cm,top=2cm,bottom=2cm]{geometry}
\usepackage{commath}
\usepackage{tabu}
\usepackage{algorithm}
\usepackage[noend]{algpseudocode}

\usepackage{amsthm}
\usepackage{hyperref}
\usepackage{mathtools}
\DeclarePairedDelimiter\ceil{\lceil}{\rceil}
\DeclarePairedDelimiter\floor{\lfloor}{\rfloor}

\title{Design Assignment 2 - CS330}

\author{
  Ginuga Saketh\      \texttt{150257}
  \and
  Arjun Sinha\      \texttt{150134}
  \and 
  Shubh Gupta\    \texttt{14670}
  \and 
  Deepak Gangwar\  \texttt{14208}\\
  Group No\.\ 11\
}

\begin{document}
\maketitle

\section{Part I}

Values of quantums used are:
\begin{itemize}
\item $\frac{1}{4}^{th}$ quantum: 32
\item $\frac{1}{2}^{th}$ quantum: 65
\item $\frac{3}{4}^{th}$ quantum: 96
\item Minimum value of quantum for maximum CPU utilization: 20
\end{itemize}
\subsection{Batch 1}

CPU utilization is as follows:

\begin{center}
\begin{tabular}{|c|c|c|}
\hline
S.No. & Scheduling Algorithm & CPU Utilization\\
\hline
1 & Non-Preemptive Default Nachos Scheduling & 56\\
2 & Non-Preemptive Shortest Burst First & 56\\
3 & Round Robin with 1/4 quantum & 72\\
4 & Round Robin with 1/2 quantum & 65\\
5 & Round Robin with 3/4 quantum & 62\\
6 & Round Robin with minimum quantum & 80\\
7 & UNIX Scheduler with 1/4 quantum & 75\\
8 & UNIX Scheduler with 1/2 quantum & 68\\
9 & UNIX Scheduler with 3/4 quantum & 64\\
10 & UNIX Scheduler with minimum quantum & 85\\
\hline
\end{tabular}
\end{center}

Average Wait Time is as follows:

\begin{center}
\begin{tabular}{|c|c|c|}
\hline
S.No. & Scheduling Algorithm & Average Wait Time\\
\hline
1 & Non-Preemptive Default Nachos Scheduling & 22190\\
2 & Non-Preemptive Shortest Burst First & 22190\\
3 & Round Robin with 1/4 quantum & 73260\\
4 & Round Robin with 1/2 quantum & 60584\\
5 & Round Robin with 3/4 quantum & 56965\\
6 & Round Robin with minimum quantum & 101678\\
7 & UNIX Scheduler with 1/4 quantum & 72009\\
8 & UNIX Scheduler with 1/2 quantum & 59091\\
9 & UNIX Scheduler with 3/4 quantum & 55334\\
10 & UNIX Scheduler with minimum quantum & 100610\\
\hline
\end{tabular}
\end{center}


\subsection{Batch 2}

CPU utilization is as follows:

\begin{center}
\begin{tabular}{|c|c|c|}
\hline
S.No. & Scheduling Algorithm & CPU Utilization\\
\hline
1 & Non-Preemptive Default Nachos Scheduling & 82\\
2 & Non-Preemptive Shortest Burst First & 82\\
3 & Round Robin with 1/4 quantum & 96\\
4 & Round Robin with 1/2 quantum & 97\\
5 & Round Robin with 3/4 quantum & 93\\
6 & Round Robin with minimum quantum & 97\\
7 & UNIX Scheduler with 1/4 quantum & 97\\
8 & UNIX Scheduler with 1/2 quantum & 96\\
9 & UNIX Scheduler with 3/4 quantum & 94\\
10 & UNIX Scheduler with minimum quantum & 97\\
\hline
\end{tabular}
\end{center}

Average Wait Time is as follows:

\begin{center}
\begin{tabular}{|c|c|c|}
\hline
S.No. & Scheduling Algorithm & Average Wait Time\\
\hline
1 & Non-Preemptive Default Nachos Scheduling & 22141\\
2 & Non-Preemptive Shortest Burst First & 22141\\
3 & Round Robin with 1/4 quantum & 74330\\
4 & Round Robin with 1/2 quantum & 60665\\
5 & Round Robin with 3/4 quantum & 57879\\
6 & Round Robin with minimum quantum & 101566\\
7 & UNIX Scheduler with 1/4 quantum & 72937\\
8 & UNIX Scheduler with 1/2 quantum & 58613\\
9 & UNIX Scheduler with 3/4 quantum & 56621\\
10 & UNIX Scheduler with minimum quantum & 100371\\
\hline
\end{tabular}
\end{center}



\subsection{Batch 3}

CPU utilization is as follows:

\begin{center}
\begin{tabular}{|c|c|c|}
\hline
S.No. & Scheduling Algorithm & CPU Utilization\\
\hline
1 & Non-Preemptive Default Nachos Scheduling & 94\\
2 & Non-Preemptive Shortest Burst First & 94\\
3 & Round Robin with 1/4 quantum & 99\\
4 & Round Robin with 1/2 quantum & 99\\
5 & Round Robin with 3/4 quantum & 99\\
6 & Round Robin with minimum quantum & 99\\
7 & UNIX Scheduler with 1/4 quantum & 99\\
8 & UNIX Scheduler with 1/2 quantum & 99\\
9 & UNIX Scheduler with 3/4 quantum & 99\\
10 & UNIX Scheduler with minimum quantum & 99\\
\hline
\end{tabular}
\end{center}

Average Wait Time is as follows:

\begin{center}
\begin{tabular}{|c|c|c|}
\hline
S.No. & Scheduling Algorithm & Average Wait Time\\
\hline
1 & Non-Preemptive Default Nachos Scheduling & 22141\\
2 & Non-Preemptive Shortest Burst First & 22141\\
3 & Round Robin with 1/4 quantum & 73517\\
4 & Round Robin with 1/2 quantum & 59578\\
5 & Round Robin with 3/4 quantum & 57971\\
6 & Round Robin with minimum quantum & 101181\\
7 & UNIX Scheduler with 1/4 quantum & 73505\\
8 & UNIX Scheduler with 1/2 quantum & 58520\\
9 & UNIX Scheduler with 3/4 quantum & 56321\\
10 & UNIX Scheduler with minimum quantum & 100167\\
\hline
\end{tabular}
\end{center}



\subsection{Batch 4}

CPU utilization is as follows:

\begin{center}
\begin{tabular}{|c|c|c|}
\hline
S.No. & Scheduling Algorithm & CPU Utilization\\
\hline
1 & Non-Preemptive Default Nachos Scheduling & 100\\
2 & Non-Preemptive Shortest Burst First & 100\\
3 & Round Robin with 1/4 quantum & 100\\
4 & Round Robin with 1/2 quantum & 100\\
5 & Round Robin with 3/4 quantum & 100\\
6 & Round Robin with minimum quantum & 100\\
7 & UNIX Scheduler with 1/4 quantum & 100\\
8 & UNIX Scheduler with 1/2 quantum & 100\\
9 & UNIX Scheduler with 3/4 quantum & 100\\
10 & UNIX Scheduler with minimum quantum & 100\\
\hline
\end{tabular}
\end{center}

Average Wait Time is as follows:

\begin{center}
\begin{tabular}{|c|c|c|}
\hline
S.No. & Scheduling Algorithm & Average Wait Time\\
\hline
1 & Non-Preemptive Default Nachos Scheduling & 33251\\
2 & Non-Preemptive Shortest Burst First & 33251\\
3 & Round Robin with 1/4 quantum & 73951\\
4 & Round Robin with 1/2 quantum & 60159\\
5 & Round Robin with 3/4 quantum & 56777\\
6 & Round Robin with minimum quantum & 101682\\
7 & UNIX Scheduler with 1/4 quantum & 73990\\
8 & UNIX Scheduler with 1/2 quantum & 60097\\
9 & UNIX Scheduler with 3/4 quantum & 56612\\
10 & UNIX Scheduler with minimum quantum & 101699\\
\hline
\end{tabular}
\end{center}

We observe that on minimizing the value of the quanta, we get maximum CPU utilization. Smaller the quanta, leads to more number of context switches. But however, this definition of CPU utilization doesn't involve work done in kernel mode, and thus we get maxed CPU utilization. For smaller value of quanta, the increase in context switches leads to a large overhead which is undesirable.

As to why smaller values of quanta leads to maximum utilization - due to the increase in the number of context switches, the CPU burst time as well as total time increases, but so does the ratio of the CPU burst time to total time (very crudely put, it is somewhat like $\frac{p+1}{q+1}>\frac{p}{q}$).
\section{Part II}

For Batch 5, the Average waiting time results are as follows:

\begin{itemize}
\item Non-Preemptive Default Nachos Scheduling Algorithm: \textbf{36482}
\item Non-Preemptive Shortest Burst First Scheduling Algorithm: \textbf{33268}
\end{itemize}

We can see that the Shortest Burst First scheduling algorithm has the smaller average waiting time (i.e smaller sum of waiting times), than the naive non-preemptive Nachos scheduling algorithm. Similar to the question in the mid-semester examination, scheduling jobs with smaller CPU bursts earlier, leads to less waiting for the other threads, as can be seen by an exchange argument (if we have 2 jobs with CPU bursts \textbf{a} and \textbf{b}, with $a<b$, then scheduling a before b leads to lesser waiting times, than scheduling b before a).

\pagebreak
\section{Part III}

The various errors in the estimation of CPU bursts are mentioned below:

\begin{center}
\begin{tabular}{|c|c|}
\hline
Batch No.& Estimate Error in CPU Burst\\
\hline
Batch 1 & 0.861970\\
Batch 2 & 0.913513\\
Batch 3 & 0.810605\\
Batch 4 & 1.000000\\
Batch 5 & 0.694081\\
\hline
\end{tabular}
\end{center}

On increasing OUTER\_BOUND, we see that the error decreases. This is because on increasing OUTER\_BOUND, the number of IO bursts, and likewise CPU bursts increase, and the exponential estimation keeps getting closer to the real estimate.

\section{Part IV}

The quantum used in the calculation of the following stats is \textbf{100}.

The statistics for differentiating between the 2 preemptive algorithms are as follows:

\begin{center}
\begin{tabular}{|c|c|c|c|c|}
\hline
Scheduling Algorithm& Maximum& Minimum& Average& Variance\\
\hline
Round Robin& 68210& 5692& 33078& 411933324\\
\hline
UNIX scheduler& 68200& 9948& 36675& 434216271\\
\hline
\end{tabular}
\end{center}

We can see that the maximum time is around same for both the algorithms. This is because the maximum time is approximately the sum of running times of all jobs (with scheduling and context switch overhead). 

The variance for the UNIX scheduler is more than for the Round Robin scheduler. This is because for the UNIX scheduler we complete jobs with values of lower priority earlier, than scheduling other jobs. So, we have jobs with lower priority completed much earlier, and then the jobs with higer priority are scheduled. Hence, the running times are far removed from the mean, and hence the variance is more. This is unlike in Round Robin, where jobs are always scheduled after some quanta have passed.

\end{document}









